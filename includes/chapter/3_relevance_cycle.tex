\chapter{Anforderungsanalyse und Problemdefinition}
\section{Problemdomäne und Use-Case-Identifikation}

Im Rahmen dieses Projekts werden mit Telemetriedaten von den Porsche LMDh-Rennwagen \textbf{BILD?} gearbeitet. Diese Telemetriedaten werden über mehrere Tausend Sensoren erfasst und liefern während Trainings- und Rennsessions ununterbrochen Messwerte, die in Echtzeit in eine Cloud-Plattform übertragen werden.\footnote{Vgl. Experteninterview 1 (29.08.2025)} Dort liegen sie als Zeitreihendaten vor und stehen Ingenieuren wahlweise direkt für Detailanalysen zur Verfügung oder werden in Form von Metriken aufbereitet. Unter Metriken versteht man statistische Kennzahlen wie den Durchschnitt, das Minimum oder Maximum über definierte Zeitabschnitte, zum Beispiel pro Runde oder pro Strecken-Sektion. Gerade diese Metriken bilden die Grundlage, auf der Performance Engineers ihre tägliche Arbeit aufbauen.\footnote{Vgl. Experteninterview 2 (12.09.2025)}
Im aktuellen Workflow prüfen Performance Engineers zunächst die Kennzahlen in Dashboards, um Auffälligkeiten zu erkennen. Das können Temperatursprünge in schnellen Kurven sein oder ungewöhnlich hoher Reifenverschleiß auf bestimmten Streckenabschnitten.\footnote{Vgl. Experteninterview 1 (29.08.2025)} Allerdings fällt auf, dass diese Auswertung fast ausschließlich manuell erfolgt. Die Ingenieure verbringen pro Rennwochenende mehrere Stunden damit, Metriken zu sichten, Trends zusammenzuführen und in Setup-Empfehlungen zu übersetzen.\footnote{Vgl. Experteninterview 1 (29.08.2025)} Das führt nicht nur zu Verzögerungen, sondern birgt auch das Risiko, subtilere Muster zu übersehen, etwa wenn ein Zusammenspiel aus Streckentemperatur, Gas- und Bremsprofil nur in Extremlagen auffällt.
Aus dieser Ausgangslage ergibt sich ein Use Case, der direkt an das beschriebene Problem anschließt: die Vorhersage der Fahrzeugbalance, gemessen als Understeer, auf Basis der vorhandenen Telemetrie-Metriken.\footnote{Vgl. Experteninterview 2 (12.09.2025)} Das Ziel ist nicht, sämtliche Sensorrohdaten in Echtzeit zu verarbeiten, sondern die bereits aggregierten Metriken zu nutzen, um eine Balance-Prognose zu erstellen. Auf diese Weise könnten Ingenieure statt langer manueller Durchsicht direkt fundierte Empfehlungen erhalten und proaktiv handeln.
Mit der Fahrzeugbalance-Vorhersage würde sich der Arbeitsablauf von reaktivem Nachjustieren hin zu vorausschauender Optimierung verschieben. Ingenieure könnten Anpassungen bereits dann vornehmen, wenn sich ein akuter Über- oder Untersteuern-Trend ankündigt.\footnote{Vgl. Experteninterview 2 (12.09.2025)} Darüber hinaus verspricht dieser Use Case eine objektivere Entscheidungsbasis: Anstelle persönlicher Einschätzungen stünden reproduzierbare Kennzahlenmodelle im Mittelpunkt. Damit würde das bestehende System von punktueller Datenansicht auf datengetriebene Automatisierung übergehen und den Zeitaufwand für Analyse sowie Setup-Änderungen deutlich verringern.

\textbf{Für längeren Text, erkläre LMDh, Cloud Plattform, Telemetriedaten, Metriken}

\section{Anforderungsableitung und -bewertung}


Aus den beiden Gesprächen mit dem Performance Engineer am 29.08.2025 und 12.09.2025 (siehe Anhang A.1) lassen sich konkrete Anforderungen an das ML-Artefakt ableiten. Die funktionalen Anforderungen betreffen in erster Linie die Fähigkeit, die Fahrzeugbalance, gemessen als aUndersteer, auf Basis vorhandener Telemetrie-Metriken zuverlässig vorherzusagen. Dieses Ziel folgt direkt aus der Erkenntnis, dass manuelle Analysen mehrere Stunden pro Rennwochenende beanspruchen und dass frühe Hinweise auf Balanceabweichungen häufig erst verspätet offensichtlich werden.

Neben der reinen Vorhersagegenauigkeit muss das Modell nachvollziehbare Ergebnisse liefern. Ein hoher Erklärungsgrad ermöglicht es den Ingenieuren, die zugrundeliegenden Einflussfaktoren zu verstehen und Entscheidungen auf einer nachvollziehbaren Basis zu treffen. Aus diesem Grund wird der Einsatz von Explainable-AI-Methoden wie SHAP festgeschrieben. Solche Methoden erscheinen erst dann sinnvoll, wenn das zugrundeliegende Regressionsmodell eine gewisse Güte erreicht hat. Empirische Studien legen nahe, dass bei lokal interpretierten Surrogatmodellen ein \(\,R^2\)-Wert von mindestens 0,85 erforderlich ist, um die Zuverlässigkeit der Erklärungen sicherzustellen\footnote{Samek, Montavon, Bach 2021, S. 12–15}.

Schließlich ist eine lückenlose Dokumentation aller Eingangsdaten, Vorverarbeitungs­schritte und Modellparameter vorgesehen. Nur so kann eine vollständige Reproduzierbarkeit gewährleistet werden, und die erstellten Prognosen bleiben validierbar.

Diese Anforderungen bilden die Grundlage für die spätere Modellarchitektur und das Evaluationsdesign in den folgenden Kapiteln.

\textbf{Bessere Begründung?}

\section{Abgrenzung des DSR-Artefakts}

\textbf{Nur für 2 Tracks, Es soll kein universelles ML Modell werden, sondern einen Proof of Concept für Porsche Motorsport, der zeigt dass es grundsätzlich möglich ist, mit ML die Fahrzeugbalance vorherzusagen. Deshalb beschränkt sich die Arbeit auf diese beiden Rennevent und nur auf die Rennsessions.}