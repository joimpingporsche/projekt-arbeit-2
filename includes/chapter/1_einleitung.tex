\chapter{Einleitung}

\textbf{Einführung Motorsport bei Porsche}

\section{Problemstellung und Motivation}
Im Rennsport hängt die Fahrzeugbalance von zahlreichen, komplex miteinander verflochtenen Einflussgrößen, sodass Ingenieure aktuell nicht nachvollziehen können, welche Kombinationen dieser Parameter eine Balance-Änderung hervorrufen, und daher erst reaktiv handeln, wenn Abweichungen auftreten. Ziel ist es, zu verstehen, wie diese Einflussgrößen gemeinsam die Fahrzeugbalance steuern, um proaktiv fundierte Entscheidungen treffen zu können.

\section{Zielsetzung und Forschungsfragen}

Ziel dieser Arbeit ist es zunächst, ein maschinelles Lernmodell zu entwickeln, das auf Basis der Parameter  
\textit{Tracktemperatur, Reifentemperatur, Reifendruck, Fahrertyp, Mechanical Balance (Anti-Roll-Bar-Front/Rear),}  
\textit{Reifenmischung, Reifenmileage, Softwareeinstellungen (Brake Balance, Traktionskontrolle)} und  
\textit{Kraftstoffgewicht} die Fahrzeugbalance mit hoher Genauigkeit vorhersagt. Darauf aufbauend wird  
mittels Explainable AI untersucht, wie diese Einflussgrößen gemeinsam die Zielvariable determinieren,  
um Ingenieuren ein Verständnis der wirkenden Zusammenhänge zu ermöglichen.

\paragraph{Forschungsfragen}
\begin{enumerate}
  \item Lässt sich ein ML-Modell konstruieren, das die Fahrzeugbalance zuverlässig prognostiziert?
  \item Können Explainable-AI-Methoden wertvolle Einblicke darin liefern, wie die betrachteten Einflussfaktoren gemeinsam die Fahrzeugbalance bestimmen?
\end{enumerate}


\section{Aufbau der Arbeit}
Die vorliegende Arbeit ist so strukturiert, dass beide Forschungsfragen systematisch beantwortet werden. 

In Kapitel 2 werden zunächst die relevanten theoretischen Grundlagen vermittelt. Abschnitt 2.1 führt in die Design-Science-Research-Methodologie ein, mit der Frage 1 (Entwicklung eines Vorhersagemodells) begleitet wird. Abschnitt 2.2 beschreibt die Grundlagen zu maschinellen Lernverfahren für Regression und insbesondere XGBoost, die später in Kapitel 4 zur Realisierung des Modells eingesetzt werden. Abschnitt 2.3 gibt einen kompakten Überblick zu Explainable-AI-Techniken, die in Kapitel 7 zur Beantwortung von Frage 2 (Erkenntnisgewinn mittels Explainable AI) relevant sind.

Kapitel 3 widmet sich der Anforderungsanalyse und Problemdefinition im DSR Relevance Cycle. Hier werden die fachlichen, funktionalen und nicht-funktionalen Anforderungen abgeleitet, die notwendig sind, um Frage 1 zu beantworten: Die Festlegung von Zielgrößen wie \(R^2\ge0{,}85\) und die Varianz-Abdeckung durch die Top-10-Features sichert die Modellqualität.

Kapitel 4 beschreibt das Design und die Implementierung des Vorhersagemodells (DSR Design \& Build Cycle). Dabei werden die Datenpipeline, das Training mit XGBoost und das Hyperparameter-Tuning detailliert dargestellt, um Frage 1 praktisch umzusetzen.

Kapitel 5 enthält die Evaluation des trainierten Modells (DSR Rigor Cycle). Quantitative Metriken wie R², RMSE und die kumulierte Varianz-Erklärung durch die Top-10-Features werden ausgewertet, um abschließend zu bestätigen, dass Frage 1 erfüllt ist.

Kapitel 6 reflektiert das Modell-Artefakt und zieht Lessons Learned aus dem Entwicklungsprozess. Hier werden Design-Entscheidungen und Einschränkungen kritisch beleuchtet, bevor der Fokus auf Frage 2 wechselt.

Kapitel 7 ist dem Erkenntnisgewinn mittels Explainable AI gewidmet. Mit SHAP-Analysen und weiteren Interpretationsmethoden wird Frage 2 beantwortet: Es wird aufgezeigt, wie die betrachteten Einflussfaktoren gemeinsam die Fahrzeugbalance bestimmen und welche praktischen Einsichten Ingenieuren damit zur Verfügung stehen.

Kapitel 8 fasst die Ergebnisse zusammen, beantwortet nochmals beide Forschungsfragen und gibt einen Ausblick auf mögliche weiterführende Arbeiten. Zudem enthält es eine kritische Selbsteinschätzung des DSR-Prozesses und des erzielten Nutzens.  
