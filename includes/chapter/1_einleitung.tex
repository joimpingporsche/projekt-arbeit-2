\chapter{Einleitung}

\section{Problemstellung}

Der Motorsport ist ein hochkompetitives Umfeld, in dem Verbesserungen der Fahrzeugperformance oft nur Bruchteile von Sekunden bringen, aber entscheidend sind. Moderne Rennfahrzeuge sind mit Tausenden von Sensoren ausgestattet, die kontinuierlich Telemetriedaten erfassen und übertragen. Diese Datenmengen ermöglichen eine beispiellose Sichtbarkeit in Fahrzeugverhalten, Streckenbedingungen und Fahrerdynamik. Trotz dieser technologischen Verfügbarkeit verlässt sich die Telemetrie-Analyse in der Praxis stark auf manuelle Prozesse: Renningenieure sichten Dashboards, identifizieren Auffälligkeiten und leiten daraus Setup-Anpassungen ab. Dies ist zeitintensiv und anfällig für Übersehungen subtiler Muster, die erst bei Kombination mehrerer Parameter sichtbar werden.
Diese Situation repräsentiert ein klassisches Problem der angewandten Informatik: Ein großes Datenvolumen, klare Geschäftsziele, aber unzureichende Automatisierung zur systematischen Mustererkennung. Machine Learning verspricht hier die Fähigkeit, aus historischen Daten Muster zu erkennen und auf neue Situationen zu generalisieren.

Die Fahrzeugbalance ist eine Schlüsselgröße für Rennfahrer und Renningenieure. Sie beschreibt die Tendenz des Fahrzeugs, in Kurvenfahrt weniger oder stärker zu lenken als vom Fahrer über das Lenkrad eingegeben. Die Fahrzeugbalance beeinflusst direkt Fahrbarkeit, Sicherheit und Renngeschwindigkeit. Die Balance hängt von zahlreichen, komplex verflochtenen Einflussfaktoren ab.
Das zentrale Problem besteht darin, dass Renningenieure diese Zusammenhänge derzeit nicht systematisch erfassen. Sie können nicht vorhersagen, welche spezifischen Parameterkombinationen zu welchen Balance-Änderungen führen. Dies zwingt sie zu reaktivem Handeln: Erst wenn Abweichungen in den Daten offensichtlich werden, werden Anpassungen vorgenommen. Ein proaktiver, vorhersagender Ansatz wäre wertvoll, da er die Fähigkeit bietet, frühzeitig zu signalisieren, dass sich die Balance-Charakteristik verschiebt und damit setup-Optimierungen proaktiv zu planen.

Damit ergibt sich die zentrale Forschungsmotivation: Können Machine-Learning-Modelle aus Telemetriedaten lernen, Fahrzeugbalance vorherzusagen? Und wenn ja, unter welchen Bedingungen generalisieren diese Modelle zuverlässig auf neue Renn-Events mit veränderten Kontexten?

\section{Zielsetzung}

Diese Arbeit verfolgt das Ziel, ein Machine-Learning-Modell zur Vorhersage der Fahrzeugbalance auf Basis aggregierter Telemetrie-Metriken zu entwickeln und systematisch zu evaluieren. Das Modell soll dazu dienen, Renningenieure zu unterstützen, indem es automatisiert Vorhersagen liefert, statt dass Ingenieure manuell Daten analysieren müssen.

Die Entwicklung folgt der \ac{DSR} Methodologie, die in Kapitel 2 eingeführt wird. Dies bedeutet konkret: (1) Systematische Anforderungsanalyse aus der Anwendungsdomäne, (2) Rigorous Build des Artefakts, (3) Umfassende Evaluierung anhand wissenschaftlicher Metriken, und (4) Reflexion von Design Knowledge für die Wissensbasis.

\section{Forschungsansatz und Aufbau der Arbeit}

Die vorliegende Arbeit strukturiert sich in sechs Kapitel, die den DSR-Prozess vom Problem zur Lösung abbilden. Kapitel 2 vermittelt die theoretischen Grundlagen, Kapitel 3 analysiert Anforderungen, Kapitel 4 dokumentiert das Design und die Entwicklung des Artefakts, Kapitel 5 evaluiert das Modell-Artefakt und Kapitel 6 schließt den DSR-Zyklus mit Reflexion und Ausblick.

Diese Arbeit verbindet technische Artefakt-Entwicklung mit methodisch fundierter 
Evaluation. So wird das entwickelte ML-Modell anhand etablierter Metriken bewertet, 
und die gewonnenen Erkenntnisse tragen zur Wissensbasis über ML-Anwendungen im 
Motorsport bei.