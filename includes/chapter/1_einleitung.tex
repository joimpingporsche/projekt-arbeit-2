\chapter{Einleitung}
Die vorliegende Arbeit adressiert ein praktisches Problem an der Schnittstelle von Motorsport-Ingenieurwesen und Datenanalyse. Telemetriedaten stehen in großem Umfang verfügbar, werden aber überwiegend manuell analysiert. Dieses Kapitel beschreibt die Notwendigkeit einer automatisierten Fahrzeugbalance-Vorhersage und etabliert die zentrale Forschungsmotivation, die diese Arbeit im Kontext der Design Science Research Methodik beantworten wird. Es bildet damit den Relevance Cycle des \ac{DSR}-Prozesses und verankert die nachfolgenden Kapitel in einer klar definierten Problemdomäne.

\section{Problemstellung}

Der Motorsport ist ein hochkompetitives Umfeld, in dem Verbesserungen der Fahrzeugperformance oft nur Bruchteile von Sekunden bringen, aber entscheidend sind. Moderne Rennfahrzeuge sind mit Tausenden von Sensoren ausgestattet, die kontinuierlich Telemetriedaten erfassen und übertragen. Diese Datenmengen ermöglichen eine beispiellose Einsicht in Fahrzeugverhalten, Streckenbedingungen und Fahrerdynamik. Trotz dieser technologischen Verfügbarkeit verlässt sich die Telemetrie-Analyse in der Praxis stark auf manuelle Prozesse: Renningenieur*innen sichten Dashboards, identifizieren Auffälligkeiten und leiten daraus Optimierungen der Rennstrategie oder des Fahrzeug-Setups ab. Dies ist zeitintensiv und anfällig für Übersehungen subtiler Muster, die erst bei Kombination mehrerer Parameter sichtbar werden.
Diese Situation repräsentiert ein klassisches Problem: Ein großes Datenvolumen, klare Geschäftsziele, aber unzureichende Automatisierung zur systematischen Mustererkennung. \ac{ML} verspricht hier die Fähigkeit, aus historischen Daten Muster zu erkennen und auf neue Situationen zu generalisieren.

Die Fahrzeugbalance ist eine Schlüsselgröße für Rennfahrer*innen und 
Renningenieur*innen. Sie wird definiert als das Verhältnis der 
Seitenführungskräfte zwischen Vorder- und Hinterachse und beschreibt damit 
das dynamische Gleichgewicht des Fahrzeugs in der Kurvenfahrt. 
Bei konstant gehaltenem Lenkwinkelinput ändert sich nicht die Lenktendenz 
des Fahrzeugs selbst, sondern resultiert vielmehr eine veränderte 
Fahrdynamik, die sich beispielsweise in unterschiedlichen Schräulaufwinkeln 
oder Gierraten manifestiert. Diese Balance beeinflusst wesentlich die 
Fahrbarkeit, Sicherheit und Rundenzeit, da sie bestimmt, wie 
responsiv und präzise das Fahrzeug auf Lenkkommandos reagiert.\footnote{Vgl. \cite{Milliken1995}, S. 53 ff.}
Die Balance hängt von zahlreichen, komplex verflochtenen Einflussfaktoren ab.
Das zentrale Problem besteht darin, dass die physikalischen Zusammenhänge 
zwischen Umgebungsparametern, indirekten Faktoren (wie Reifentemperatur und 
Reifenabnutzung) und der resultierenden Fahrzeugbalance zu komplex sind, um 
sie analytisch zu modellieren. Obwohl alle diese Größen kontinuierlich live 
telemetrisch messbar sind, können Renningenieur*innen daraus nicht intuitiv 
ableiten, wie sich verändernde Umgebungsbedingungen und Reifenzustände auf 
die Balance auswirken werden. Sie können nicht vorhersagen, welche spezifische 
Kombinationen aus Umgebungsparametern und indirekten Faktoren zu welchen 
Balance-Änderungen führen und damit nicht proaktiv reagieren.

Dies zwingt sie oft zu reaktivem Handeln: Erst wenn Abweichungen in den Messdaten 
offensichtlich werden, werden Anpassungen vorgenommen. Ein proaktiver Ansatz 
basierend auf Vorhersagen wäre wertvoll, da er die Fähigkeit bietet, frühzeitig 
zu signalisieren, dass sich die Balance-Charakteristik verschiebt und damit 
Setup-Optimierungen vorzeitig zu planen.

Damit ergibt sich die zentrale Forschungsmotivation: Können ML-Modelle aus 
beobachteten Mustern in gemessenen Telemetriedaten lernen, die 
Fahrzeugbalance vorherzusagen, ohne die zugrundeliegenden physikalischen 
Kausalitäten explizit modellieren zu müssen? Und wenn ja, unter welchen 
Bedingungen generalisieren diese Modelle zuverlässig auf neue Renn-Events 
mit veränderten Kontexten? Die daraus folgenden Forschungsfragen werden 
in Kapitel 3.4 präzisiert.

\section{Zielsetzung}

Diese Arbeit verfolgt das Ziel, ein \ac{ML}-Modell zur Vorhersage der Fahrzeugbalance auf Basis aggregierter Telemetrie-Metriken zu entwickeln und systematisch zu evaluieren. Das Modell soll dazu dienen, Renningenieur*innen zu unterstützen, indem es automatisiert Vorhersagen liefert, anstatt dass Ingenieur*innen manuell Daten analysieren müssen.

Die Entwicklung folgt der \ac{DSR} Methodik, die in Kapitel 2 eingeführt wird. Dies bedeutet konkret: (1) Systematische Anforderungsanalyse aus der Anwendungsdomäne, (2) Rigorous Build des Artefakts, (3) Umfassende Evaluierung anhand wissenschaftlicher Metriken, und (4) Reflexion von Design Knowledge für die Wissensbasis.\footnote{Vgl. \cite{Hevner2004}, S. 82 ff.}

\section{Forschungsansatz und Aufbau der Arbeit}

Die vorliegende Arbeit strukturiert sich in sechs Kapitel, die den \ac{DSR}-Prozess vom Problem zur Lösung abbilden. Kapitel 2 vermittelt die theoretischen Grundlagen, Kapitel 3 analysiert Anforderungen, Kapitel 4 dokumentiert das Design und die Entwicklung des Artefakts, Kapitel 5 evaluiert das Modell-Artefakt und Kapitel 6 schließt den \ac{DSR}-Zyklus mit Reflexion und Ausblick.

Diese Arbeit verbindet technische Artefakt-Entwicklung mit methodisch fundierter 
Evaluation. So wird das entwickelte \ac{ML}-Modell anhand etablierter Metriken bewertet
und die gewonnenen Erkenntnisse tragen zur Wissensbasis über \ac{ML}-Anwendungen im 
Motorsport bei.