\chapter{Theoretische Grundlagen}
\section{Design Science Research Methodologie}
\section{Experteninterviews}

Im Rahmen der Relevance Cycle-Phase wurden zwei informelle Gespräche mit einem Performance Engineer vom Porsche Motorsport geführt, um praxisnahe Anforderungen an die Telemetriedatenanalyse zu ermitteln. Das erste Gespräch fand am 29.08.2025, das zweite am 12.09.2025 statt.

Ziel der Unterhaltungen war es,  
\begin{itemize}
  \item die derzeitige Nutzung von Telemetrie-Metriken in Dashboards nachzuvollziehen,  
  \item zentrale Einflussfaktoren auf die Fahrzeugbalance aus Ingenieurssicht zu identifizieren und  
  \item Erwartungen an ein automatisiertes Vorhersagemodell hinsichtlich Genauigkeit und Erklärbarkeit zu skizzieren.  
\end{itemize}

Obwohl auf einen festen Leitfaden verzichtet wurde, entstanden klare Hinweise darauf, dass die manuelle Sichtung von Metriken mehrere Stunden pro Rennwochenende beansprucht und komplexe Zusammenhänge zwischen Parametern selten systematisch erfasst werden. Diese Erkenntnisse legen den Grundstein für die Anforderungsableitung in Kapitel 3.  


Die vollständige Transkription ist im ~\ref{subsec:transkript-meeting1} verfügbar.




\section{Maschinelle Lernverfahren für Regression}
\section{Modellinterpretation und Explainable AI}