\anhangteil{Rohtranskripte der Experteninterviews}

\subsubsection{Erstes Meeting mit Performance-Ingenieur (29.08.2025)}
\label{subsec:transkript-meeting1}


Interviewer: 
Meeting mit Ingenieur, 29.08.2025 11 Uhr. 


Interviewer: 
Ja, freut mich, vielen Dank für deine Zeit.  
Jetzt wollte ich dich fragen, ob du dich ganz kurz vorstellen könntest.  

Ingenieur: 
So ja, kann ich kurz machen, ich bin schon länger bei Porsche.   Teamleiter  von der LMDH Performance Truppe.   seit LMP Formel E. Jetzt LMDh, was wir machen, Performance, vor  allem bei der Entwicklung, Simulationslastik, Kennwerte vorgeben für die anderen Abteilungen, auch, der Drag brauchen wir, wie viel Abtrieb brauchen auf dem Auto.    Auch für die Reifenentwicklung mit Michelin zusammen, Rundenzeitberechnung machen wir, auch die Modellierung dazu. Und eben wenn des Auto mal fährt, Datenanalyse. Aber auch im Teil der performancerelevanten Software im Auto kommt auch von uns. Also, Funktionskontrolle. 

Interviewer: 
Okay, sehr cool. Vielleicht stelle ich jetzt noch mal ganz kurz das vor, was wir vorhaben. Ich bin jetzt wie gesagt bei EMO 6 für nächsten drei Monate und muss von meiner Hochschule aus eine wissenschaftliche Arbeit schreiben über das Projekt, das ich mache. Und das Projekt, das ich mache, ist nämlich...  Wir wollten, oder das wurde von, das kam von EM6 und von Paul, das Thema hoch, dass man vielleicht mit einem Proof of Concept starten möchte, um zu gucken, gibt es denn Auswertungen, Analysen, die man mittels der, also aus den Telemetriedaten ziehen kann, irgendwelche Aussagen, da rauskommen, die sich, auf die man ein Machine Learning Modell trainieren kann und einfach um mal zu gucken, ist es möglich, es da  Daten und einfach mal ein Proof of Concept zu starten.  Ja, genau. Das ist jetzt noch relativ am Anfang, deswegen als auch unser Termin. Ich bin neu im Motorsport, ich  schaue auch mal zu Formel 1, aber das hilft mir dann jetzt in der Tiefe der Thematik nicht sehr viel weiter. ist eine falsche Serie. Falsche Serie, mich geht es jetzt grundsätzlich mal ganz kurz darum, bisschen das Verständnis zu bekommen. Was  macht ein Performance-Ingenieur, also du hast es gerade schon mal kurz angeschnitten, vielleicht auch mal bisschen, also wirklich vielleicht so von oben kommen.  Genau,  was ist deine Jobrolle oder die Jobrolle des Performance Engineers und dann vielleicht auch eine Einordnung, wie denn sowas an einem Wochenende aussieht, wenn du das vorhin schon gemeint hast, also Datenanalyse, das ist ja da wahrscheinlich eher die Richtung, die für mich eher interessant wäre. Genau, also vielleicht könntest du mal bisschen so top down, mal bisschen grundlegend mal sagen, wie das so abläuft.  


Ingenieur: 
Ich kann erst mal starten, es hilft. Performance-Ingenieur, Performance, die sind verantwortlich für die Performance vom Auto. Performance heißt, möglichst optimal das Auto auf der Strecke einzusetzen. Optimal bezüglich Rundenzeit logischerweise.  Das ist ein Thema. Und das zweite ist, was sie noch machen ist, sie gucken auch, dass das Auto sicher betrieben wird. Das heißt, sie schauen in Betrieb, zum Beispiel in die Bremstemperatur.  zu stark ansteigt und kritisch  dann heben sie die Hand und sagen, wir haben da Problem. Diskutieren dann mit Renningenieur zusammen, h olt das Auto rein und dann wird es repariert.  sie gucken, z.B. wenn die Fahrhöhe vorne zu niedrig ist, anders als erwartet, dann muss man auch die Hand heben und sagen, da passiert irgendwas nicht. Oder wenn das Bremspedal zu lang wird, weil Verschleiß oh  zu hoch ist. sind auch so sicherheitsrelevante Themen, was wir anschauen.  Idealer Weise tritt das nicht auf. Dann ist Ihre Aufgabe halt Setuparbeit, am Auto, du kannst ja ganz viel einstellen, viel mehr als beim Straßenauto. Du kannst deine Federn, Steifigkeiten ändern. Da gibt es ganz viele Optionen. Es gibt verschiedene Aero-Konfigurationen. Die Fahrwerke kannst du einstellen.   verschiedene Federkombinationen in Serie schalten, dass du verschiedene Charakteristiker hast und noch viel mehr. Das ist Ihre Aufgabe, das während des Rennwochenendes bis zum Rennen zu optimieren.   Zusammen mit dem Engenieur und natürlich auch mit dem Fahrer. Der Fahrer fährt ja nicht nur das Auto, nur im Kreis. Sondern nach jeder Änderung kriegt man Feedback von dem Fahrer, ob das ins Gute oder ins Schlecht war. Aus seiner Sicht und das fließt in die Entscheidung ein, was man als nächstes ändert am Auto.   Was man ändert am Auto, wie gesagt Fahrradfeedback ist wichtig und eben auch die Daten dazu.  Das Telemetrie oder Kabeldaten. WEC hat leider sehr wenig Telemetrie, per Reglement ist das vorgegeben. IMSA hat da mehr, da hat man Probleme, mit der Logging rate. Im Endeffekt, man guckt an Daten, macht die auf, schaut sich die Balance an vom Auto.  Balance heißt dass das Auto unter oder übersteuert. du schaust deine Fahrhöhen an, ob das Auto zu tief ist aufgesetzt oder zu hoch ist und du noch Potenzial hast tiefer zu gehen, je tiefer du bist, desto mehr Abtrieb hast du, was damit auch Rundenzeit bringt. Das ist so die Aufgabe von einem Performance Engineer. das Auto mit Renningenieur und Fahrer zu optimieren. Sie sind eine Verantwortung für das Setup vom Auto. Verstehe. 

Interviewer: 
Danke für die Einführung. Das war jetzt auf jeden Fall schon mal sehr hilfreich.  Kannst du mir noch einen kurzen Umriss geben, wie dann das Zusammenspiel an dem Wochenende vielleicht aussieht.  Ich stelle mir das gerade aktuell so vor, man hat die Fahrt in Teilen links.  Es werden Daten gesammelt, die Autos sind auf der Strecke, wie du schon beschrieben hast. Dann gibt es bei euch mehrere Performance Engineers, die sich die Daten angucken und wahrscheinlich auf verschiedene Dinge  achten auf Basis ihrer Erfahrungen oder wie das Team eingeteilt ist.   

Ingenieur: 
Der Performance Ingenieur, der hat Tools dazu auch,  der simuliert, wenn ich jetzt die Feder ändere, dann ändert das in der Theorie, folgendes an meinem Auto, das ist tiefer, höher, ich sehe auch die Abtriebwerte theoretisch, das gibt mir alles mein Simulationsmodell her. Der gleicht das dann ab mit Streckendaten.  Und je nachdem vom Feedback vom Fahrer, wenn der sagt, hey, ich zu viel Untersteuern, dann weiß der Performance-Engineer. Wenn ich an dem Parameter drehe, vom Auto, würde sich das ändern. . Dann würde sich die Verlauf vom Auto in die Richtung ändern. Normalerweise simuliert er das vorhandene Tool mit HH, also das ist unsere Datenbank.  kann er sagen ich ende jetzt mein Setup, die Fahrhöhre, dann wird das simuliert und dann kriegt er die Werte aus. Das macht er nicht erst am Wochenende, wo er sich das frei überlegt. Der macht sich davor ganz viele Gedanken, der legt sich so einen Blumenstrauß an Setups zusammen, weil er diskutiert ist schon mit seinem Renningenieur. auch auf Erfahrung von den Jahren davor, was funktioniert hat, was sie da gefahren haben. Wir haben auch den Fahrsimulator, das ist erste Tool. Auf dem Papier mit verschiedenen Setup Optionen. Du fährst dann am Fahrsimulaltor, du suchst jetzt schon mal aus, was gut funktioniert, was nicht so gut funktioniert, runter weg. Und mit den Optionen, Simulation ist eine Simulation, muss man noch validieren an der Strecke, gehst an die Strecke und fährst die eventuell dann einfach gegen. Validierst dann deine Theorie und nimmst dann das Beste ins Rennen mit, beziehungsweise ins Qualifying, z.B. um Unterschiede etc.  Das ist der Prozess von der Performance , der definierte Setup. Da es die Mechaniker, die ich ausdrucken soll, die haben ja Tablet. Da sehen Sie, ich muss die Feder ändern, muss hier die Fahrhöhe anheben  und dann stelle ich das am Auto ein. Da gibt es auch wieder von der Struktur einen Nr1 mechaniker.  Der kriegt  die Info und verteilt es dann an sein Team am Auto. Und dann wird das umgesetzt, was  der Performance Engineer zusammen mit dem Renngenieur entschieden hat. Genau, dann wird das umgesetzt, dann fahren die, kriegst die Daten und kannst dann sozusagen interativ und etc. über dich finden. 

Interviewer: 
Okay, ja, also danke, ja perfekt, danke für den Überblick. Das schärft jetzt ein bisschen mein Bild jetzt auch, weil ich halt eben neu bin in der Thematik, wie das generell funktioniert. Genau, aber dann würde ich jetzt mal ein bisschen erklären, was wir, also wie und was wir machen wollen. Also...  Der Ansatz ist ziemlich Greenfield. haben dieses Konzept,  wir mit Machine Learning, also Modell, auf Telemetriedaten trainieren wollen. Das ist die Baseline. Von hier aus können wir jetzt viele verschiedene Wege gehen. Ich muss nicht in irgendeine Richtung gehen. Ich kann verschiedene Dinge anwenden.  Ganz grundlegend, was könnte man  im Machine Learning gibt es dann verschiedene Teilbereiche. Der erste Teilbereich sind Klassifikationsmodelle. Man bekommt ja immer bei einem Machine Learning ganz viele Inputs, zum Beispiel in unserem Fall die verschiedensten Telemetrie-Channels. Das können 1, 2 bis 100. 100 sein, die man dort hineingibt und im Optimalfall kommt hinten ein Output aus.  Und dieser Output kann je nachdem was man möchte, kann ja verschieden sein. Da gibt es zum einen das Klassifikationsmodell, das ist das erste, dann kommt hinten, man gibt seine Telemetriedaten rein und dann kommt zum Beispiel hinten   Ich habe mir paar Beispiele ausgesucht auf Motorsport, unabhängig davon, ob das  Sinn macht. Zum Verständnis ist, man z.B. in der TdMT-Daten von den Trainings reinhaut und hinten kommt dann eine Reifenmischung aus, was vielleicht Sinn macht, Soft, Medium, Hard. Also  eine Klasse, Soft, Medium oder Hard. Sowas  in die Richtung. Das ist ein Klassifikationsmodell. man sagt, Fahrverhalten klassifizieren.  wie schnell wird Gaspedal, Bremspedal, wie schnell wird am Lenkrad gedreht und hinten kommt aus, dass es aggressives Fahrverhalten, mittleres Fahrverhalten oder ähm...   optimales Fahrverhalten. Also in die Richtung kann man das machen. Man kann die Output natürlich selber definieren.  wäre das erste. Das zweite, was eventuell interessant ist für das, was wir machen wollen, sind die Vorhersage- oder Regressionsmodelle. Das heißt, man trainiert das Modell mit vielen verschiedenen Telemetriedaten, die relevant sein könnten, um einen Wert, hätte,   vorhersagen. Also zum Beispiel in der Rundenzeit vorhersage ich.  gibt dem Modell ganz ganz viele verschiedene Telemetriedaten, es aktuell ist. Und dann fahre ich meinen ersten, zweiten Sektor, dritten Sektor und es wird immer aktualisiert, zum Beispiel die prognostizierte Endrundenzeit vorhergesagt zum Beispiel. Das wäre was für ein Regensohnsmodell. Also ich will kurz einfach vermitteln, wie das funktionieren könnte. Man mappt von ganz vielen verschiedenen Inputs auf einen Output. Wie der Output aussieht,  können verschiedene Sachen sein.  die Sachen wären so die Hauptkategorien,  die ich mich fokussieren würde. Also man kann einmal eine Klassifikation, also es kommt dann ein String aus, also kein numerischer Wert oder Regression, man versucht wirklich numerische Werte vorherzusagen auf Basis von anderen Parametern. Genau. Das ist so das, was es in der Theorie kann.  Was für das Machine Learning Modell wichtig ist, ist, mit was für Daten es trainiert wird.  Für das Taining braucht man beide Enden  von dem. Man braucht den Input und den Output. Den Input, den Telemetriedaten, die ganzen Channels, speist man ein und gibt aber dann  den wahren Output mit dem, was es sein soll, was hinten auskommen soll. Das Machine Learning Modell, lernt während des Tainings,  wie es von den verschiedensten Inputs zu diesem  einen  Output mappt. ganz grobe Abriss, vielleicht dass du ein bisschen, ich weiß nicht genau wie technisch du vielleicht  auch schon in der Thematik, aber das wollte ich noch mal ganz kurz abgerissen haben, dass du vielleicht ein bisschen einschätzen kannst in was für eine Richtung  es gehen könnte. Also im Optimalfall wäre es für mich, also jetzt am Beispiel der Rundenzeit Vorhersage, da hatte ich mir schon ein bisschen reingeguckt. meine, das nennt sich Label, der Output ist ein Label für das Training und das ist ja relativ einfach zu bekommen dieses Label. Man nimmt einfach von jeder Lab die letzte Laptime. Dann hat man das Label und speist verschiedenste Parameter rein, die Geschwindigkeit des Autos, erste Runde, Sektor 2, zweite Sektorzeit, Bremsdruck. alle möglichen Channels könnt man reingeben und das Modell findet selber heraus während des Trainings wie wichtig welcher Parameter ist und wie die zusammenhängen um eben den beschriebenen Output B auszugeben am Ende. Genau das wäre jetzt Beispiel Rundenzeitvorhersage. Und da wäre es mir wichtig, weil wir jetzt noch keine hochkomplexe Sache machen möchten, dass man sagt, und weil ihr  sehr viel beschäftigt seid, dass man eben diesen Output B auch bereits entweder berechnen kann oder näherungsweise berechnen kann oder bereits wie die Lap-Time aus den Daten ersichtlich sind.  Verstehst du grob? 

Ingenieur: 
Ja, also zu dem Thema Rundenzeit haben wir die Predigten laptime, schon im Auto, die sind ziemlich gut, die ist ein relativ Simpler Ansatz, also nichts mit machine learning. Gewisserweise merkt sich nur das Auto die Referenzrundenzeit, was der Fahrer aktuell gefahren ist und vergleicht die aktuelle Runde zu dieser Referenzrundenzeit.  Es dann eben eine Predictive Post, also bin ich drunter. Ich habe es selber nicht geschrieben, es funktioniert aber ziemlich gut.  Das heißt, du kommst auch ziemlich gut da rein, wo  wo der Fahrrad tatsächlich dann landet. Also klar weißt du nicht, ob er jetzt in drei Kurven einen Fehler macht.  Also das kann keiner vorher sagen. Das funktioniert ganz gut. Nur um es anzumerken, dass wir in der Richtung laufen, dass  du was machst, was es schon gibt, gut funktioniert, da könnte man vielleicht was machen, was es noch nicht gibt. Die Sache ist, wir generieren sehr viele Daten, auch KPIs. Das Schwierige ist, das Ganze in kurzer Zeit, weil es ist ja wirklich nicht viel Zeit, um das zu verstehen, die Zusammenhänge zu verstehen.  Genau, wenn ich jetzt mal ein Beispiel mache.  Du fährst mit einem Auto, Temperatur ist x von der Strecke, die hat einen Einfluss auf das Verhalten vom Reifen, wie viel Grill du hast, wie  viel Luftdruck der Regeningenieur gerade reingemacht hat, wie du gefahren bist mit dem Auto, wie  du den Reifen aufgewärmt hast in den ersten fünf Runden, hat einen Effekt auf  den Grip, du in Runde 20 haben wirst. Ob du aggressiv angefangen hast, hat einen Einfluss ob du länger schnell fahren kannst oder halt weniger. Es sind so schwierige Entscheidungen oder auch, jetzt haben wir verschiedene Compounds Soft Medium Hart, welcher ist denn jetzt gerade der optimale? Temperatur bei 30 Grad oder bei 40 Grad und ich erwarte vielleicht, dass ich im Renn 50 Grad fahre, kann ich dann noch die Medium fahren oder muss ich den harten schon fahren, die eigentlich für die Bedingungen gedacht ist? Das sind so interessante Fragen. Es ist jetzt weniger mit Telemetrie, sondern eher, du sammelst das Wochenende über Daten und versuchst die dann zu interpretieren, möglichst schnell um die richtigen Entscheidungen vor dem Rennen zu treffen.  Während man fährt, was immer ganz interessant fände, wenn die Fahrer sagen, sie hätten gerne mehr Feedback oder Guidance auch?  und die Telemetrie haben. Beispiel im Auto, kannst so Sachen verstellen, das ist ja nicht so gegeben, wie du rumfährst, gerade bei den Systemen. Ja, deine Traktionskontrolle, also wie viel Moment kriegt der Fahrer für einen bestimmten Schlupf den du am Rad siehst.  Auch mit dem Hinblick auch wieder Verschleiß. Was macht mein Auto in Runde 20? Wenn  ich komplett viel Schlupf generiere, dann baut man bei es irgendwann ab. ist es geschickter am Anfang bisschen konservativer auch mit den Settings zu fahren, dass das Auto eben nicht permanent ausbricht an der Hinterachse.  Und den Fahrer das als Hinweis schon mitzugeben. Auf Basis der Telemetriedaten  zu sagen, geh mal mit deinem Setting bisschen konservativer, weil wir sehen in den Daten, du überfährst den Reifen gerade. Da gibt es auch verschiedene Kanäle, wie den Schlupf, man angucken kann, die Temperatur von den Reifen.  selber sein Fahrstil, wie viel Energie bringt er mit seinem Fahrstil in die Reife  Es gibt verschiedene  auch Kanäle, wie gesagt, auf Telemetrie, die man an anschauen fand. Und entweder live, so machen wir es jetzt, weil wir so Erfahrung wollen, du guckst die Daten meistens ein halbes Jahr lang an und dann weißt wann du ungefähr, was in welcher Richtung, einstellen muss.   So gibt man den Fahrrad Hinweise, ob er was ändern sollte oder nicht. Dann sagt er manchmal, hey voll gut, danke.  Manchmal passt es halt nicht, aber dann muss man halt weiter lernen.

Interviewer: 
Das klingt sehr interessant. Man würde zum Beispiel aus ganz vielen verschiedenen Channels den Output generieren, überfahren, nicht überfahren oder so in die Richtung.  Okay, ja das klingt auf jeden Fall nach einem sehr interessanten Thema, weil das ist eben auch genau so was, wo ich so den Hauptbenefit sehe. Ich meine, ihr könnt ja alle, ich meine, ich brauche ja jetzt kein Modell darauf trainieren, wenn du siehst, Tire Temperature größer 50,  dann passiert irgendwas, ja. Genau,  das macht ja keinen Sinn. Aber eben sowas wie du beschreibst,  was man über eine längere Zeit... über eine längere Zeit einen Wert beobachten muss und dann auf Basis von Erfahrung keine... Also da  stehen bestimmt intrinsische Regeln dahinter. Aber nichts, was der Mensch so ausdrücken könnte, in einem Code zum Beispiel, sondern eben wie du sagst, ist viel Erfahrung und Bauchgefühl  von Jahren. Ich glaube, das wäre auf jeden Fall ein Gebiet, wo so ein Modell großen Potenzial hätte.  


Ingenieur: 
Ich glaube, das würde helfen. Man muss ja gucken, dass das, glaube ich, deine Arbeit jetzt nicht so ein Konzept sein man versucht, die Weltformel zu generieren. sondern vielleicht mal was Simples, einen simplen Ansatz. 
Das ist meine Reifentemperatur, habe ich auf Telemetrie und das ist meine Balance vom Auto, habe ich auf Telemetrie und den Schlupf auch. Das einzige was mir bei dem Thema einfällt, das Thema, also das Training, die Trainingsdaten zu generieren. Da würden wir jetzt, also weiß ich... Vielleicht kennst du dann den Ansatz, wie man sagt, die Daten müssen gelabelt werden. Ich habe alle Telemetrie-Daten,  pro Sekunde oder zu 100 Hertz reinkommen.  


Interviewer: 
Ich müsste markieren, hier diesem Zeitpunkt wurde der Reifen überfahren. Ich muss dieses überfahren Label irgendwie setzen, im besten Fall automatisiert. Das ist das Einzige, was wir bei dem Thema vielleicht bisschen ...  Kopfzerbrechen oder was ein bisschen kompliziert ist.  Ich könnte es jetzt ja, wie du sagst, als Basis auf Erfahrung. Ich kann jetzt ja gerade nicht einfach mir die Telemetät anzeigen und sagen, hier wurde der Reifen zu überfahren, hier, hier und hier.  Das klingt nachher ein super cooles Thema, aber im besten Fall müssten wir es irgendwie schaffen, ihr mir das vermitteln könnt, dass ich diese Daten labeln kann. Weil sonst die Konsequenz wäre, wenn das nicht klappt, wie man es dann machen müsste, dass jemand, der diese Erfahrung im Kopf hat, ein Tool an die Hand und die Telemetriedaten markieren muss, hier markieren, überfahren, hier überfahren, hier überfahren und das am besten 1000 mal. Aber das ist natürlich für euch nicht praktikabel.   


Ingenieur: 
Das gibt Kennwerte, die wir haben von hier sagen, zum Beispiel...   Da gehen wir dann ein bisschen von der Telemetrie weg. Das sind dann wirklich Kennwerte. Das ist die Telemetrie und Kabeldaten. der Post-Programm-Testing. Und das ist dann die Integrale, mit laufenden irgendwelchen Mathefunktionen dahinter. Die Auswertung machen, da kommt dann eine Zahl raus. Zum Beispiel, wenn man sagt, man schaut sich Anzahl der Snaps an, in der Runde, wie häufig bricht das Heck aus, wenn der Fahrer ins Gas geht. Wenn der Reifen aufgeht, dann passiert das häufiger, als wenn der Reifen komplett neu ist.  Das wäre zu sagen, ein Master für die ZIG-Datensätze, die ich habe. ich das als Eingang in den Fahrstil, Energie, was weiß ich. Und irgendwann beim Fahrer A habe ich mehr Snaps ab Runde 10 und bei Fahrer B ist es erst ab Runde 20. Ich weiß nicht ob.  Auf Basis dessen könnte man jetzt hier überlegen Fahrer b, besser gemacht, ja. Weil bei dem kommt die Stab ein bisschen später. Ein anderes Kriterium, was eigentlich auch super simpel wäre, eigentlich könnte man mal sagen, Rundenzeit. Ich kann mir das mal deshalb ganz kürzlich erst einmal... 
Das ist einfach ein Long Run, das heißt wir fahren viele Runden von zwei verschiedenen Setups , gleicher Fahrer. Du siehst hier die Rundenzeit. Das Auto wird dann immer langsamer, je mehr Runden du fährst. Also kriegen wir weniger Grip. Hier sieht man, wo der Reifen dann wirklich abkackt. wird die Rundzeit dann langsamer. Also das kannst du in X-Richtungen verschieben.  


Interviewer:
Ah, ja. Das klingt cool. Warte mal, wenn ich mir überlege...   Man gibt es sozusagen anhand von aktuellen Fahrverhalten aus, solange man in eine Lebensdauer prognostiziert.   Also wenn du sagst nicht überfahren nicht überfahren sondern du sagst also so wie du jetzt in den letzten in den letzten zwei Sektoren oder der letzten Lab gefahren bist würde der Reifen jetzt noch drei halten und dann fährt er ein bisschen langsamer oder entspannter und dann dann springt der Wert hoch auf fünf oder sowas. Das hört sich gut an. 


Ingenieur:
Ja, das ist halt jetzt, hier waren zwei Setups, das kann ja verschiedene Gründe haben, warum das passiert. Das kann jetzt sein, wenn ich das nur die Rundenzeit dem Modell gebe, das weiß er nicht. Dann weiß erhalt nicht, waren das jetzt zwei Fahrer und der eine ist einfach aggressiver gefahren, der andere ein bisschen schonender. Oder wie in dem Fall sind es zwei Setups, das war der gleiche Fahrer, der versucht gleich zu fahren. Oder waren das irgendwelche Einstellungen am Lenkrad, die er unterschiedlich gemacht hat.  Ich glaube, muss man dem Modell auch mitgeben. War das jetzt ein anderer Fahrer, war das ein anderer Fahrstil? Da kommt das Thema wieder dazu. Wie viel Energie steckt der Fahrer in die Reifen? Wie fährt er? oder was sind die Einstellungen gewesen. Aber ich glaube, geht erstmal proff of concept. Da könnte man noch sagen, du nimmst mal Daten von Le Mans oder von irgendeinem Dauerlauf, wo das Setup einfach gleich geblieben ist. Du fährst einfach 24 Stunden lang. Dann kann man das schon mal aus xen, dass sich da irgendwas am Setup tut, dann ist es wirklich nur Fahrerunterschiede oder Fahrerunterschiede. 


Interviewer:
Das ist eine sehr gute Idee, glaube ich. Das wäre glaube ich echt mal für so Proof of Concept eine gute Idee, dass man erst mal abgekapselt, wie du sagst, nur für Lés mans, nur für ein Setup, guckt, wie sich das entwickelt. Das könnte klappen. Da muss ich mir angucken, wie die Daten, ich meine jetzt 24 Stunden, da kommt natürlich eine ordentliche Datenmenge herum, ob das erreicht für ein Training. Aber das ist auf jeden Fall schon mal eine gute Richtung, in die du mich da glaube ich schickst. Das ist auf jeden Fall eine gute Richtung. Da haben wir auf jeden Fall eine Kripp-Pasei. Wir haben auch in den Daten, was wir machen, auf Basis der Grabedaten trainieren wir ein Modell oder fitten ein Reifenmodell, wo wir dann wissen, der Reifen hat jetzt weniger Grip. Er baut dann ab. Du fittest jede Runde ein Reifenmodell. Und dein Krippparameter fällt dann ab, einfach damit das zur Runde passt. Und da wir ganz viele Ausgärtungen auch dazu.  Das klingt nach  einem guten  Startpunkt für mich. Okay, also jetzt mit Blick auf die Uhr, weil du meinst, hast jetzt auch... Ja, genau. Ich würde jetzt auf jeden Fall mal das Hausaufgabe für mich mitnehmen. Das ist jetzt echt ein cooles Thema. Ich würde mich da ein bisschen einarbeiten, mir ein paar Gedanken dazu machen. Ich bin nächste Woche im Urlaub. Und dann würde ich dir vielleicht in zwei Wochen, falls du da da bist, einen Termin einstellen, wenn das okay wäre für dich.  Dann bringe ich noch mal paar fische Gedanken rein. Du kannst mir dann noch da drauf bisschen Input geben. Wäre das okay für dich? 

Ingenieur:
Ja, ich glaube, gut, wenn du für das Meeting so bisschen aufmalst, wie du dir das vorstellst. Hier kommen Daten, da passiert das, da passiert das, das kommt raus. 

Interviewer:
Vielen Dank Fabian für deine Zeit. war wirklich sehr gut. Danke.

\subsubsection{Zweites Meeting mit Performance-Ingenieur (12.09.2025)}
\label{subsec:transkript-meeting2}


Interviewer:
Zweites Meeting mit Ingenieur am 12.09.2025 um  10.30 Uhr.

Ingenieur:
Hi, guten Morgen, grüß Dich. 

Interviewer:
Freut mich, dass wir es wieder schaffen, hier zusammen zu finden. Genau, gleich vorweg, ich habe auf jeden Fall die Woche ziemlich Gas gegeben und war die ganze Woche eigentlich, hatte ich mich mit dem ursprünglichen Thema beschäftigt, in die Richtung, du mich gelenkt hast, was auch wirklich...  ultra interessant ist und das war das das Reifendegrationsthema. Da habe ich mich die ganze Woche auch schon durch die Daten gewühlt und habe mir so bisschen auch schon eingeguckt wie man es machen könnte. Und grundlegend ist mir ein, größeres Problem aufgefallen und zwar gibt es  eben wie ich vielleicht auch vorher schon mal gesagt hatte, das Thema, dass man das labeln muss. Also man muss sich irgendwelche Regeln, also euer Bauchgefühl, also wirklich greifbar machen und definieren, wo man Schwellwerte setzt von verschiedenen Parametern, die man dann erreicht oder so weiter, um dann zu sagen, ja jetzt hier der Reifen gerutscht, hier nicht gerutscht, dass man eben nicht manuell in die Daten reingehen muss und das wirklich manuell machen muss. Das ist ein größeres Hauptproblem und ich glaube, das hatte ich letztes Mal gar nicht erwähnt. Mein Projekt hier ist nur bis November und ich muss auch noch eine wissenschaftliche Arbeit in diesem Zeitraum bis November darüber schreiben. Das heißt, ich habe relativ wenig Zeit. Und dementsprechend war ich gestern recht glücklich, weil mir ist dann noch eine Idee gekommen. Ich habe nochmal darüber nachgedacht, was du gesagt hast. Und eine der Sachen, die du angesprochen hast, ist, dass die Fahrer oft sagen, dass sie gerne mehr Feedback hätten. Dass sie genauer oder noch mehr Feedback hätten. Und da ist mir eine Idee gekommen,  kombiniert, wie man das gleich umsetzen könnte, ist, dass man ein  Modell trainiert, um Fahrerfeedback zu generieren. Und die Idee, die ich dabei hatte, ist,  Man nimmt die Telemetriedaten und man hat ja für jedes Outing  schon in den Daten gesetzt welcher Fahrer gerade fährt. Und sozusagen man trainiert  das Modell darauf auf verschiedene Fahrer und dann gibt es eine Technologie die nennt sich Explainability AI. Das heißt normalerweise kennt man es ja, dass solche AI Programme ziemliche Black Boxen sind.  bringt irgendwas rein, die trainiert sie und dann gibt man Input und es kommt irgendwie ein Output raus. Wunder, Wunder. Aber da gibt es tatsächlich auch schon Ansätze, um eben genau diese Black Box aufzubrechen und zu gucken, wie kommt die KI zu der Entscheidung. Und hier kommt der Mehrwert ins Spiel. Meine Idee jetzt aktuell, wer man guckt, also man trainiert sie auf die verschiedenen Fahrer und kann zum Beispiel dann auch live gucken.  In dem und dem Sektor war Fahrer 1, 2 oder 3 schneller und dann kann man  wirklich relativ tief in die Daten eintauchen und schauen  wo die Unterschiede bei den Fahrern liegen über die AI. Das war jetzt meine Idee. Das wäre aus einem aus  dem großen, also für mich wäre es deutlich angenehmer umzusetzen.  Aus dem Grund, ich die Labels, also welcher Fahrer gerade fährt, einfach aus den Daten ziehen kann. dementsprechend, also dieses Labeling-Thema ist bei so Maschinen-Learning-Projekten meistens immer der größte Painpoint und der größte Faktor, warum was scheitert. Und da sehe ich das jetzt eben auch eventuell bei dem iPhone-Digaktionsthema, weil es schon eben was Großes ist und wenn es so einfach wäre, dann wird es wahrscheinlich schon gemacht werden. Dementsprechend hätte ich jetzt...  


Ingenieur:
Ich kann dir kurz mal zeigen, was ich mir gedacht hätte. Okay. Und dann kannst du mir sagen, ob das so kompliziert ist. Okay, okay. Und dann können wir vielleicht noch mal in das andere einsteigen, weil ich hatte mir jetzt auch kurz vor dem Meeting noch mal fünf, zehn Minuten Zeit genommen. Das ist super nett. Ich mir paar Gedanken gemacht. Also, ja, geht ein bisschen in die Richtung, aber was mich halt, was halt echt interessant wäre...  Oder was uns immer umtreibt ist, was macht die Fahrzeug Balance? ist das Auto eher neutral oder untersteuernd? Und das hängt halt ab von ganz vielen Parametern. Daher ist es für uns oder als Mensch relativ schwierig zu verstehen, was jetzt was beeinflusst. So das Gleiche gilt für den Grip. Den kann man mal außen vor lassen.  Deswegen haben wir gesagt, wir gucken mal erst mal Renndaten an, weil Renndaten, da kann ich die ganze Setup Arbeit, was der Renningenieur macht und was noch das Ganze noch viel komplizierter macht, erst mal außen vor lassen. Wenn man sich so bisschen vorstellt wie eine Gleichung, das wäre mein Y, die Fahrzeug Balance, die hängt halt von ganz vielen Eingangswerten X und dann gibt es hier eine Funktion, die wir nicht kennen. Und dann kommt da die Balance raus.  Die Eingangsparameter, zum Beispiel die Streckentemperatur. Die Strecke jetzt, die messen wir ja. Wir haben eine Wetterstation, das gibt es in der Datenplattformen auf jeden Fall. Gibt es in unseren Workbooks. Die gibt es. Reifentemperatur haben wir, die messen wir. Reifendruck haben wir, messen wir auch. Wie du sagt Fahrer, wissen wir auch, welcher im Auto sitzt. Wir wissen die Knöpfe an denen der Fahrer dreht, was er einstellen kann. Der kann nämlich die Stabis verstellen. Da gibt es auch einfach Knöpfe 1, 2, 3, 4, 5. Recht diskrete Stufen. könnte auch direkt die mechanische Balance angucken, aber das geht vielleicht auch mit den Stabis. Weiß nicht ob man hier Aero-Balance braucht. Die ändert sich eigentlich nicht. Die wäre auch recht konstant. Reifenspec ähnlich wie Fahrer, wissen wir ob ein trocken Reifen oder ein Regenreifen am Auto ist und trocken Reifen wissen wir sogar welcher Gummimischung, ob es ein Soft, Medium oder Hard ist, stellt der Fahrer ein. Dann wichtig ist natürlich auch die Mileage. Wie viel Laufzeit hat der Reifen schon? Da bin ich mir unsicher, ob wir das schon in der Datenplattformen. Gibt es aber genauso schon als in  HH. Genauso wie die Streckentemperatur müsste man die Mileage einlesen können. Jetzt weiß ich nicht, wer das macht. ob das ein Monin oder ein Paul machen kann. jeden Fall wäre das so was, wo ich sie irgendwann mal gerne hätte. Dass man die Mileage von dem Set hat. Dann Software. Da du natürlich auch einiges verstellen. Deine TC, also Traktionskontrolle oder deine Bremsblance. Aber das sind auch Knöpfe. Da dreht er dran. Also TC 1, 2, 3, 4, Auch diskrete Stufen. und dann natürlich der Sprit, den es gibt. Ist das Auto schwer oder leicht? Das waren für mich mal die Schnelle, die Prior 1 Faktoren. Hier können wir mal Aero ausklammern. Das bleibt hoffentlich auch konstant, weil wir ja Setup nicht ändern.  So und von den Ausgangsgrößen, habe hier mal eine Runde aufgemacht, nicht eine Runde, sondern ein Workbook in Power BI. ist jetzt ein Auto Le Mans, komplette Rennen. Man kann es vielleicht ein bisschen anders darstellen. Wir haben ja schon Kennwerte.  Dann siehst du hier, wir wollen die Balance ganz gut.

Interviewer:
ist es das, was ihr euch live an der Strecke anguckt, das PowerBi Dashboard, oder habt ihr Wintax mit den genauen Telemetriedaten offen? Das sind ja die Metriken, oder? 

Ingenieur:
Beides, also Wintax sind wirklich deine Telemetriedaten als Datenstrom und dann haben wir halt Auswertung. kannst Gates definieren. das, du sagst, wo du die Thresholds brauchst. Und dann wo du sagst, du kannst das Cluster. Das machen wir hier. Okay, wir haben einen Cluster Entry, also Eingang der Kurve, Mitte der Kurve und Kurvenausgang. Und dazu die Balance. Und das für jede Runde gibt es dann einen Wert. Das ist ein Mittelwert über alle Kurven. Eingang. Mid-Corner und Exit und dann siehst du übers komplette Rennen, ihr Blau ist Kevin Estrid, dann haben wir Laurence und  den Matt Campbell. Hier gibt es einen Wert. Der ändert sich schon ein bisschen das kann jetzt halt die Balance abhängig sein von Streckentemperatur von ihren Settings was weiß ich hat sich auch hier während des stints ändert sich die Balance, das ist wahrscheinlich effekt von mileage oder auch fuel load das ist halt die Sache ich weiß es halt einfach nicht.


Interviewer:
Was genau sagt die Balance? Wie das Gewicht verteilt ist über die Reifen? Ist das die Balance?

Ingenieur:
Die Balance die die Fahrer beschreiben, die Fahrzeug Balance. Hast du viel untersteuern im Auto? Im Endeffekt, was mathematisch dahinter steckt. Du hast zwei Signale. Es ist vereinfacht. Zwei Signale. Den Lenkwinkel. Also wie viel lenkt der Fahrer? Und die Gearrate. Das Auto misst wie schnell sich das Auto dreht. Dann kannst du dir umrechnen, die Gierrate in einem Lenkwinkel. Wenn der Fahrer lenkt und das Auto reagiert sofort genauso wie der Fahrer lenkt, hast du ein sehr neutrales Auto. Indem du die Bewegung vom Auto mit dem Input vom Fahrer vergleichst. Wenn der Fahrer jetzt extrem viel lenken muss, aber das Auto bewegt sich gar nicht, typischen Straßenautos, dann hast du sehr viel Untersteuern. Also der Fahrer lenkt sehr viel. Aber das Auto dreht sich einfach nicht. Wenn hingegen, wenn du, wenn der Fahrer das Lenkrad ein bisschen bewegt und das Auto dreht sich sofort, dann hast du ein sehr übersteuerndes Auto. So und das ist mit Balance gemeint, mit Fahrzeug Balance. Wie viel sozusagen das Verhältnis von  Input vom Fahrer und wie rotiert das Auto? Also wie verhält sich das Auto dann darauf? So kannst du es einteilen in, man kann es entweder ganz einfach klustern in, wenn du dann Delta rechnest, Gierrate und Lenkwinkel, wenn das Auto sozusagen mehr dreht als der Fahrer eingibt, kannst du einfach sagen übersteuernd. Wenn es geht sich genauso verhält, neutral, wenn du sehr viel Lenkwinkel brauchst, kannst du sagen untersteuernd. Kannst du sagen in die drei Gruppen. Oder was wir halt einfach hier ganz stumpf haben, ist den Wert  sozusagen. Das müsste das Delta sein von wie viel rotierte das Auto und wie viel mehr muss der Fahrer lenken als Lenkwinkel. Einfach die Differenz zwischen beiden als Mittelwert pro Runde.  Genau und das Interessante ist jetzt, das ist mein Y, aber ich verstehe nicht, warum geht das jetzt, warum reduziert sich die Balance, warum wird das Auto neutraler, also weniger untersteuernd. Woran liegt das? Liegt das jetzt dran, weil der Fahrer was verstellt hat oder weil   sich die Streckentemperatur geändert hat oder weil der Reifen ja, recht viel Kilometer drauf hat oder weil der Tank leer ist und sich dadurch die Balance ein bisschen verändert. Weiß ich nicht. Das wäre halt was uns am Endeffekt interessiert oder mich die Frage, die ich mir ganz oft stelle, wenn man den Zusammenhang weiß. Ich weiß vor allem den Einfluss von dem Parameter von der Streckentemperatur auf meine Balance. Dann weiß ich halt schon, weißt du, du fährst dann in deinem FP1 am Morgen und bist eigentlich recht glücklich mit der Balance. So, dann fährst du dein FP2 am Nachmittag. Nur da ist die Strecke 20 Grad heißer. Der Fahrer sagt auf einmal, die Balance ist komplett daneben. So, dann fängst du wieder von vorne an. Stellst dein Auto wieder ein, dass der Fahrer glücklich ist. Denn Rennen ist aber dann vielleicht wieder zu deinen Temperaturen vom FP1 und dann passt es halt wieder nicht. Du bist eigentlich immer hinten dran und tunst hinterher. Aber wenn du wüsstest, okay, ich kenne ja die Wettervorhersage ich weiß, im FP2 bin ich vielleicht 20 Grad heißer. Sagt der Fahrer, die Balance ist scheiße, mach mal was. Dann kann ich ihm sagen, ja, das liegt an der Strecke. Keine Sorge, im Rennen wird es so und so warm. Das passt alles. Oder ich kann es ihm einfach erklären auch und sagen, da ist nichts kaputt am Auto, liegt einfach an der Strecke. Die ist halt extrem viel heißer. 

Interviewer:
Verstehe. Genau, heißt also man hätte hier in dem Fall also  das Labeling wieder zurück zum Thema. Hätte man das schon? Das ist genau der Graphie eigentlich oder nicht? 
Ingenieur:
Genau. Das  weiß ich jetzt nicht, ob es im Detail, also das musst du jetzt mir sagen, das weiß ich nicht, ob das die Arbeit schon einfacher macht. 

Interviewer:
Aber das ist ja Prinzip nochmal eine neue Problemstellung. Das hat erstmal nicht viel mit dem Reifendegrationsvorhersage zu tun, sondern es ist jetzt sozusagen...  Also es ist ja ähnlich wie das andere mit dem Fahrer, bloß wir  geben Inputvariablen und haben das Labeling Output und jetzt wollen wir verstehen, was dazwischen passiert. 

Ingenieur:
Genau. 

Interviewer:
Okay. Das ist nochmal eine ganz andere Fragestellung.   

Ingenieur:
Es geht halt auch um das Richtung Reif.  Das Thema mit dem Verschleiß, den Kennwert, den haben wir noch nicht. Den werden wir auch nicht vor November haben. Da habe ich mit den Kollegen nochmal gesprochen.  Also bringt ja da nichts.  

Interviewer: 
Was soll da kommen? Ich  glaube, das habe ich jetzt noch gar nicht gehört. 

Ingenieur: 
Wir wollten ja mit der Degradation was machen. Wie baut der Reifen ab? So, und da muss ein Kennwert berechnet werden, wie hier, mit der Balance. Den gibt's aber noch nicht. Der Plan war, dass wir den vor zwei Wochen implementieren. Das kriegen sie aber nicht hin. Bis Ende November. Scheinbar komplizierter. Die wollen sagen, der Reifen ist jetzt so so viel degradiert, Genau. am Endeffekt ist, was dahinter steckt, es gibt ein Reifenmodell.   Du modellierst einen Reifen mit mechanischem Verhalten über Kennlinien, über Parameter. Wir wissen, wie er sich neu verhält, wie die Parameter ausschauen. dann kannst du jede Runde hast du die Messdaten und optimierst die Parameter so, dass dieses Modell dazu passt. Wenn das Sinn macht.  Du hast sozusagen Optimierer, der die Parameter tunen, damit mein Modell zu dem Messdaten passt. Und diese Parameter, die beschreiben  zum Beispiel den Grip. Das ist ganz einfach. Ich kann es auch ganz einfach machen. Maximale Querbeschleunigung beschreibt meinen Grip. So, wenn ich jetzt weniger Grip habe, dann geht meine maximale Querbeschleunigung runter. Das wäre ein ganz einfacher Ansatz. Du guckst ja per Runde an, was meine maximale Querbeschleunigung habe ein Modell. So Querbeschleunigung ist ein bisschen wie mechanisch mit einem Reibmodell. hast eine Last Fz. Ich kann meinen Fy berechnen. ich messe meinen Fz. Das weiß ich im Auto.  Das messe ich. Ich weiß meinen Fy. Das ist meine Querkraft, meine Querbeschleunigung. Und dann kann ich einfach ausreden, was ist denn mein Reibwert, mein myh. Mit der FZ und dem Mühe kriege ich diese Seitenkraft. Und so kann ich mir das ausreden. Das ist sehr, sehr, sehr, sehr, sehr vereinfacht. Da haben wir ein komplexeres Modell, aber so funktioniert das. Das wollen wir in die Datenplattform reinbringen. Diesen Prozess, den gibt es schon, aber erst einmal nur in Matlab.  recht viel händisch, Daten runterladen, konvertieren, Modell identifizieren. Und dann gibt es dann eine Routine und die wollen wir halt in die Datenplattform bringen. Das dauert aber noch bisschen scheinbar aufwendiger als gedacht, würde ich mal vermuten, ohne dass von den Kollegen gehört zu haben. Aber sonst wäre es ja schon drin. Das heißt also davon bin ich ausgegangen. Das heißt vor dem Urlaub hatte ich  Info, dass wir das rein kriegen.  Also in der Woche, wo du im Urlaub bist, dann hätten wir das jetzt schon, aber das gibt es nicht.  Die haben gesagt, das wird nichts bis  Ende November. Das bringt dir erstmal nichts. 

Interviewer:
Okay, das ist schon mal interessant. gut. Also wenn es das gäbe, das würde natürlich das Reifendegrationsprojekt vereinfachen. 

Ingenieur:
Ja, aber im Endeffekt ist es nichts anderes wie das hier, weil es ist einfach nur ein anderer Parameter, anderes Y dann. Also du willst ja so eine Art Prinzipstudie machen. Und mein Verständnis, wir sagen können, hey, das funktioniert prinzipiell der Workflow, dann kann ich ja alles Mögliche reinwerfen und rauskriegen. 

Interviewer:
Ja, also wo man die genau weiß, wie die Einflussgrößen das Outcome  bestimmen. Genau. Okay, dann können wir noch mal ganz kurz ein bisschen tiefer einsteigen in die Car Balance, in das Thema hier. Genau, also wir haben jetzt hier unser Zielvariable, die Car-Balance, die ist gespeichert und definiert und die können wir angucken. 
Ingenieur:
Die wird gemessen.

Interviewer:
Genau, du hattest jetzt gerade noch das andere Notebook offen, wo du schon mal versucht hast, die Inputparameter zu bestimmen.

Ingenieur:
Also hier, ist dein Y, das sind genau die drei Graphen.  Und als Eingang, das kann man auch anschauen, das kann man auch sagen, die sind ein bisschen verstreut leider. in der Darstellung. Zum Beispiel Reifen.   Das wäre zum Beispiel eine Eingangsgröße. Das ist der Front, also vordere Anti-Roll Bar also dein Stabilisator. Vorne die Einstellung vom Fahrer, was er ausgewählt hat. jetzt wieder hier hingeht, wäre der Punkt hier. Damit kann er die Balance im Auto verstellen. Dann haben wir hier die TC-Settings.  Das ist ein Software-Ding. Da sind wir hier auf. weiß nicht, ob es jetzt angezeigt wird. Dem Wert hat dann ist er nach oben gegangen, wieder runter, recht wenig verstellt. Das wäre das zweite Setting. Ja, man die Bremsen auch wieder Software, die Bremsbalance, wie er es verstellt hat. Dann haben wir noch zwei weitere für die Bremsbalance. Also eigentlich die drei Parameter Brems-Balance. Das wäre also sag mal, mit dem Block hier, hätte man eigentlich schon. Das hier wären die Outputs. Ja, so jetzt kann man es noch ein bisschen Clustern . Ich weiß nicht, ob man so kompliziert schon werden muss. Das ist noch mal ein bisschen anderes Gating, wenn man das also das ist ja gesagt die Balance, ich gesagt hatte erst mal. Eingang Mitte Exit als Mittelwert über die Runde, dann kannst du es noch ein bisschen komplizierter machen, indem du noch mal die Fahrzeug Geschwindigkeit unterscheidest in High Speed, Medium Speed, Low Speed.  

Interviewer:
Also würde zum Beispiel für einen POC, also ich bräuchte ja eigentlich eine Zielvariable, ich ja gucken, zum Beispiel Highspeed, Midcorner, oder? 

Ingenieur:
Ja, oder du schaust einfach mal Midcorner über die Runde an, ob sich da irgendwas tut. So, hier hätten wir schon die Temperaturen. Da bin ich jetzt nicht sicher, welche wir davon messen im Auto die Oberfläche vom Reifen, aber wir messen auch... 

Interviewer:
Aber da würdest du auch sagen, dass  es Sinn macht, zum Beispiel das jetzt nur auf die diesjährigen Les Mans Daten von einem Auto zu trainieren, oder?  Weil sich dann zum Beispiel nicht so Sachen, die wir jetzt nicht abbilden können, so Setup-Changes, die eben... 

Ingenieur:
Ich würde es pro Auto machen, weil die Setups unterschiedlich sind. Jetzt kannst du aber sagen, ich trainiere dreimal ein Modell, nämlich zu jedem Auto. Oder du könntest sagen, weil dieses das Ganze sollte universell ja gelten. Sozusagen wenn man dann in Differenzen oder Delta das überlegt, wenn ich sage, meine Strecke wird 10 Grad heißer, dann sollte sich sollte das Auto  übersteuernder werden. Das sagt mein Training vom Auto 6. Dann könnte ich gucken, hey, jetzt schaue ich mal, wie sich die Streckentemperatur vom Auto 5 geändert hat. Mach sozusagen Replay von meinem Modell und schau, wie gut passt mein Modell zu dem, was ich vorhersagen würde, wie gut passt es zu dem, was Auto 5 gemacht hat. Sozusagen Validierungsschleife dann oder was mein Auto 4 gemacht hat. Man könnte auch sagen, ich schmeiße alle Daten in einen Topf und schaue, ob es generelle Regeln gibt. Ja,  die Setups sind unterschiedlich, aber die Balanceänderung zu meinem Setup sollte ähnlich sein.  Ich ändere das nicht im Rennen, das Setup von allen drei Autos.  Ich glaube, man könnte erst mal sagen, du hast ja echt viele Daten. Man startet mal mit einem Auto. Ich glaube, das ist dann auch nicht viel Stress, wenn man sagt, man schmeißt dann alle drei Autos mal rein oder wie gesagt, spannend wäre ja zu sagen, ich drehe mir mal ein Auto und gucke, wie gut passt meine Vorhersage zu den anderen Autos.  Das ist jetzt die Reifentemperatur, vorne links, vorne rechts, hinten links, hinten rechts. Das sind ist Durchschnittswerte pro Runde.

Interviewer: 
Ok, verstehe. Ja das ist doch auf jeden Fall auch schon mal cool. Weil diese Durchschnitte würde man eh berechnen. Das heißt, wäre vielleicht sogar was, was ich mir direkt runterziehen könnte, ohne große Vorverarbeitung. Einfach versuchen könnte, das zu trainieren. Das Einzige, was mir hier auffällt, es gibt so ein paar Ausreißer. Sind es  Fehlmessungen hier? 

Ingenieur:
Nee, das kann es sein. Zum Beispiel, wenn du einen Safety Car hast, dann  ...  steckst nicht viel Energie in die Reifen, weil fährst nicht schnell um die Kurve, dann geht halt die Temperatur runter. Oder beim Reifenwechsel neue Reifen, wir fahren ja nicht mit Heizdecken oder mit warmen Reifen los, so wie in Formel 1, die sind ja von kalt und dann werden die mit der Zeit dann wärmer, bis es sich irgendwann stabilisiert haben. Verstehe. Das passiert hier immer.  Reifenwechsel, Start ist von kalt, dann wird er wieder heiß. Dann Boxenstopp wird aufgetankt, kühlt ein bisschen ab. Oder hier waren Safety Car, nein wahrscheinlich Boxenstopp. Kühlt ein bisschen ab, dann wird er wieder heiß. Dann haben wir wieder Reifen gewechselt. Genau, geht dann wieder runter hier. Wahrscheinlich haben sie ein bisschen Energie gespart oder waren Safety Car oder eine Slowzone. Ich weiß es nicht, da müsste man jetzt ins Detail gucken. Aber das sind die Schwankungen hier sind normalerweise irgendwie Safety Car oder Reifenwechsel. Warum das hier nach oben abrauscht, das kann ich dir nicht gar nicht sagen. Vielleicht ist da irgendein Sensor ausgefallen. Diese Surface, die sind immer ein bisschen, die gehen öfters mal kaputt. Okay. Die sind halt, das sind so Infrarot Sensoren. Die sind im Radkasten, auf dem Reifen und wenn da ein Gummifurzel abfliegt, an diesen Sensor ran. Dann kann es sein, dass der kaputt geht. Vielleicht nimmst du da einfach diese Inner-Liner. Das sind Sensoren, die vom Rad auf den Gummi gucken. Das geht nicht so schnell kaputt. Weil hier hat man auch so ein paar Ausreise. Da kannst du sagen, alles was Also 300 Grad, da schmilzt der Reifen. Das lässt sich mit dem Filter dann echt einfach.  

Interviewer:
Du sagst, also wir kriegen eigentlich alle Daten, du jetzt in deinem OneNote hast, kriegen wir hier als Metric irgendwo her, oder?  

Ingenieur:
Wir haben hier alle Sachen, außer die Mileage. Aber die wäre schon wichtig.  Das müsste man Paul fragen, dass sie das noch integrieren.  wäre auch einfach ein Wert pro Runde. Wenn es dir einfach machen willst, kannst du sagen, du nimmst die Anzahl der Runden. Dann weißt du das einzige, du halt da nicht weißt ob das ein neuer ist, der drauf gekommen ist oder ob der schon drauf war. 

Interviewer:
Wo liegt das an? Also es liegt ja aktuell ab, hast du gesagt, irgendwo, 

Ingenieur:
Ja, zum Beispiel, also ich hab hier grad HH, das ist unsere Datenbank Software. Die sind schon verlinkt. Also das Workbook. Und die Software, du siehst hier zum Beispiel ist Tracktemperature die Zieht sich die Datenplattform von dem Tool hier von HH. Genauso wie die ganze Setup-Information, steht da steht alles hier drin. So, wenn man jetzt mal guckt.   Mein Gedanke wäre, den Workflow gibt es schon. Jetzt muss nur einer sagen, ich integriert anstatt Streckentemperatur die Reifenmilage, weil ich die Kilometer vom Reifen hier schon habe und mache da einen Metric, einen unten Endwert oder Startwert, ist dann auch nicht so dramatisch und nehme das dann als Kernwert.  


Interviewer:
Könntest du, würdest du dir was ausmachen, wenn du  deine OneNote-Page shares? 

Ingenieur:
Ja, die kann ich dir dann schicken. 

Interviewer: 
Danke schön.  Ja, aber genau, das ist dann auch ein Thema, ich Paul gebe. Und ansonsten, es wäre praktisch,  ich alle Input- und alle Output-Daten an einem Ort habe, dann trainiere ich und Es ist auch wieder genau, dann geht es wieder zurück zu diesem Explainability.  Ich muss mal gucken, inwiefern, wie genau ich sagen kann. was auf jeden Fall geht, ist, ich kann schauen, welche Einflussgröße, wie stark beeinflusst und so weiter. Aber das geht auch relativ tief. 

Ingenieur:
Das hat jetzt halt nichts mehr mit Rohdaten zu tun, weil wir sind ja mal gekommen von den ersten Ideen auf Telemetrie-Daten was zu machen. Das hat dann halt nichts mehr mit Roh-Telemetrie-Daten zu tun, sondern nur noch mit Process-Daten, die schon aufbereitet wurden gewisserweise. 

Interviewer:
Das ist okay. 

Ingenieur:
So, jetzt kann man mal gucken hier. Das ist eine grafische Darstellung. Hier siehst du, da sind neue Reifen drauf gekommen.  Hier wurden die dann, das ist keine vier Punkte, die wurden dann drauf gelassen. Erster Run, logischerweise neue Reifen. Man sieht dann hier, das sind Anzahl der Runden, aber das wäre vielleicht auch ausreichend. Weil es gibt diese Zahl, du siehst hier der Reifen.  in dem Outing war von 0 Runden bis 12. Wenn ich jetzt die nächste Runde anschaue, dann siehst du, der startet bei 12 und geht dann bis 25. Das heißt, ist frisch drauf gekommen von frischen Reifen. Also hier ganz am Anfang frisch, genau. Aber der ist dann nicht mehr frisch, der hat schon 12 Runden. Und der nächste, hat dann, siehst du ja, der startet bei 25 Runden und den hat in Summe 37 drauf gehabt. Dann haben wir Fahrer gewechselt. Der Reifen hat schon eine Runde drauf gehabt. Der ist dann weitergefahren, nicht getauscht, hat in Summe 40 Runden gefahren. Dann Fahrerwechsel, frische Reifen, startet wieder bei Null. Der ist dann 13 Runden gefahren und so weiter. mein Punkt ist, es gibt hier schon einen...  Parameter Signal das dir sagt wie viele Runden hat der Reifen. 


Interviewer:
Also hast du so ein Wert ab wann von deinem Bauchgefühl ist die Peak Performance von dem Reifen wann und wie viele Runden lässt die nach? 

Ingenieur:
Ja, also Peak sind die ersten acht Runden, wenn sie pushen. Aber oft fahren sie die, also wenn sie es geschickt machen, nutzen sie die auch nicht. Weil wenn du wirklich die Peak Performance nutzt, dann machst du den Reifen ein bisschen kaputt. Das tut dir dann später eher weh. Weil er der dann schneller abbaut. Das ist ein anderes Thema. Aber ja, siehst, dass der Peak irgendwo hier im ersten Run. dann muss man gucken, der baut dann schon ab. Auto wird dann auch langsamer.


Interviewer:
Allgemein gibt es mehrere Themen für euch, wo ihr sagt, wir haben hier eine metric. Wir wissen auch wahrscheinlich welche Eingangsvariablen es gibt, aber wir wissen nicht genau, wie die zusammenhängen, um dieses Ergebnis zu produzieren, oder? Also das gäbe es für Degradation, jetzt eben für das Balance Thema und so weiter. Also da gibt es so ein paar Sachen, wo das so ist, oder? 

Ingenieur:
Ja, schon. Also es gibt verschiedene Ansätze. Du kannst natürlich sagen, ich bilde mir ein Modell von dem Reifen, nehme die ganzen Daten und fit das. Und dann weiß ich auch, wenn ich an der Streckentemperatur, Dreh, weil sich auch was passiert. Aber das ist auch extrem viel Aufwand, dann Modell zu generieren, was halt passt in allen Bedingungen. Oder der andere Ansatz wäre halt einfach jetzt. Ich habe unendlich viele Daten. Wir haben ja extrem viele Daten in dem Projekt. Und ich nehme einfach die Daten, wie sie sind, und versuche da die Zusammenhänge zu verstehen. Und  da gibt es viele Sachen, wie jetzt  also das Thema mit der Balance oder wie gesagt mit der. Degradation, wenn wir das mal drin haben. Da kann man sich unzählige Sachen vorstellen. Beispiel gibt es auch irgendeine Abhängigkeit für meine Aerobalance, die sich über die Laufzeit Da kann man kreativ werden. Deswegen, wenn man dann einen Prozess hat, wo man sagt, das ist mein X und das ist mein Y, ich trainiere das.  kann dann vorhersagen, was passiert mit meinem Y, wenn ich an X ein was ändere. Das wäre halt schon echt hilfreich. 

Interviewer: 
Dann danke für das Gespräch, ich glaube es hat sehr geholfen.

\subsubsection{Methodische Anmerkungen zu den Interviews}

Die Interviews wurden als semi-strukturierte Experteninterviews geführt und digital aufgezeichnet. Die vorliegenden Transkripte sind Rohtranskripte, die zur besseren Lesbarkeit minimal geglättet wurden, jedoch den originalen Gesprächsinhalt und -verlauf authentisch wiedergeben.

Die Gespräche dienten der:
\begin{itemize}
    \item Anforderungsanalyse für das Machine Learning Projekt
    \item Identifikation relevanter Telemetriedaten und Kenngrößen
    \item Bewertung verschiedener Ansätze (Reifendegradation vs. Car Balance)
    \item Klärung technischer Umsetzbarkeit und Datenverfügbarkeit
\end{itemize}